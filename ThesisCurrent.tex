\documentclass{article}
\usepackage[utf8]{inputenc}
\usepackage{graphicx}
\usepackage{float}
\usepackage{amsmath}
\usepackage{ragged2e}

\begin{titlepage}
 \begin{center}
        \vspace{1cm}
            
        \Huge
        \textbf{Stability Regimes of Shallow Water Waves Under General Elastic Boundaries}
            
        \vspace{1cm}
        \LARGE
        Physics 4999E - Honours Research Project
            
        \vspace{2cm}
            
        \textbf{James Wing-Chee Graham}
            

        \vspace{0.6cm} 
        Supervisor: Dr. Olga Trichtchenko\\
        
        
            
        \vspace{0.8cm}
            
 \end{center}           
\end{titlepage}
 
\usepackage[a4paper, total={7in, 8in}]{geometry}

\begin{document}
\setlength{\parindent}{0pt}

\section{Abstract}

Dispersive Hamiltonian PDE's are an important class of problems due to their extensive applications in physics. An example of such a problem is the water wave problem, describing an incompressible, irrotational fluid using the Euler equations. Stability regimes of flexural-gravity waves have yet to be fully classified, with regimes determined only for certain solutions with certain parameters [\ref{olga1}]. We sought to fully classify the linear stability regime for travelling wave solutions to the general water wave problem, that is, either waves moving under an elastic boundary with surface-flexural rigidity D, or capillary-gravity waves with surface tension \(\sigma\). Depending on chosen parameters of the system, the water wave problem can physically describe water waves propagating under an ice sheet, or it can describe blood flow through an elastic vessel. For this analysis, we generalized the approach by Trichtchenko for identifying stability regimes of dispersive Hamiltonian PDE’s [\ref{reduced}]. We found that instabilities are governed by the choice of values for D for flexural-gravity waves, and that varying \(\sigma\) for capillary-gravity waves had no effect on these high-frequency instabilities. Various regimes of stability were found in both ice thickness ranges, including a bifurcation in the centimetre range, as well as regimes of total stability and instability in the millimeter range with resonance effects that may contribute to some of the instabilities. Potential extensions of this research include a generalization of the model to waves propagating in dimensions greater than 1D, as well as the numerical computation of travelling wave solutions to investigate changes in amplitude in various stability regimes. 


\clearpage 


\section{Introduction}



\subsection{Motivation}

The mathematical basis for the modelling of a dynamical system in physics can be reduced to the solutions of partial differential equations (PDE’s) that govern the dynamics of the system. Concerning solutions of a PDE, the notion of stability describes the degree to which small perturbations of initial conditions to solutions of a given PDE generate small changes in the perturbed solution trajectories. If small perturbations of initial conditions to solutions of a given PDE generate large changes in the perturbed solution trajectories, then the PDE solution is said to be unstable. Stability theory is a section of mathematics concerning dynamical systems that has major applications in analyzing the stability of solutions for physics-based PDE models of water wave phenomena [\ref{stabilitypde}]. \\

Stability theory is integral to the study of differential equations and dynamical systems, and is a fast growing area of research from both pure and applied mathematical perspectives. The evaluation of stability regimes for various classes of PDE’s is integral to characterizing the accuracy and physical significance of solutions in space and time. Stability regimes for model equations for water waves have yet to be completely characterized, and such characterizations will also inform us about the kind of water wave phenomena that can physically exist in the real world, as well as providing useful information about water wave models for both the field of physics and environmental applied science. With the rise of global warming, weather conditions have become increasingly volatile around the world, and the accuracy of models in the field of hydrodynamic forecasting can greatly benefit from a complete mathematical characterization of stability of water waves. While unstable wave solutions are less likely to be observed in the real world, they tend to grow without bound, and thus can break the ice surface at critical amplitudes. Through restricting our research to the stability regimes of a specific class of water wave equations, we seek to further the knowledge of the field with a characterization of stability for shallow water, surface flexural-gravity waves. These waves can be visualized below in Figure (\ref{icewaves}). \\

\begin{figure}[hbt!] 
\centering
\includegraphics[scale = .4]{icewaves.jpg}
\caption{ \emph{Water waves propagating under a sheet of ice. [\ref{icewavespic}].}}\label{icewaves}
\end{figure}


 As far as the terms and scope of the study, we seek to investigate the stability of small amplitude plane wave solutions to a set of partial differential equations describing the dynamics of shallow water, periodic flexural-gravity waves under ice. Upon investigating various regimes of the system by varying the physical parameters of our equation of motion, we seek to understand the stability regimes in both the centimetre and millimetre ice thickness ranges. The centimetre range describes the water-wave ice problem, whereas the millimetre range physically represents that of fluid flow along a small elastic boundary, such as a small biological membrane, or blood vessel as depicted in Figure (\ref{bloodvessel}).  \\
 
\begin{figure}[hbt!]
\centering
\includegraphics[scale = .45]{bloodvessel.jpg}
\caption{ \emph{Diagram of coronary arteries. The average thicknesses of the left and right coronary arteries lie in the range of 3-6 mm, exactly within our millimeter thickness regime in our stability analysis [\ref{heart}]. }}  \label{bloodvessel}
\end{figure}
 

 We restrict our focus to that of scalar, dispersive Hamiltonian PDE’s of order 2N+1, \(N \epsilon Z\) with one spatial and time dimension for simplicity. The constraint of a Hamiltonian system necessitates conservation of energy, whereas the odd order constraint necessitates the dispersiveness of the system, that is, the lack of even, dissipative terms. Due to the lack of such terms, dispersive systems are generally more difficult to characterize the stability of, thereby providing a rationale for focusing on the aforementioned class of equations. \\
 
 \clearpage
 
 \subsection{Background and Objectives}
 
  In this section, we seek to present a set of conditions on our system such that we can determine stability of wave solutions. A physical description of our system is given below in Figure (\ref{wavesindeed}). In the Eulerian picture, the solution of this system amounts to determining the fluid velocity vector field \(u(x,z,t)\) that evolves in time and space according to Euler's incompressible equations, employing the appropriate boundary conditions. 
  
  
  \subsubsection{Equations of Motion, Boundary Conditions, and Dispersion Relation}
  
  
  Firstly, a general form of Euler's equations which express conservation of mass and conservation of momentum are, respectively:
  
  \begin{equation} \label{mass}
  \nabla \cdot u = 0  \\
\end{equation}

\begin{equation} \label{eq1}
  \frac{D u}{D t} = -\frac{1}{\rho}\nabla p 
  -g \hat{\boldsymbol{z}}  \\
  \end{equation}

Where \(\frac{D u}{D t}\) denotes the advective derivative, which represents the time rate of change of the velocity field at the moving location of the field. Furthermore, \(\rho\) is the density of the fluid and \(g = 9.8 \frac{m}{s^2}\) denotes gravity. We assume our fluid is ideal, meaning that it is incompressible, not subject to viscous forces, and has constant density. Physical water waves can be approximated as ideal due to the negligible effects of these assumptions. We also assume that our fluid is irrotational, and thus the vorticity of the fluid vanishes. The vorticity \( \omega \) of our fluid is defined as: \\


 
 \begin{equation} \label{eqn3}
  \omega = \nabla \times u = 0  \\
\end{equation}
  
This implies \(u\) is a conservative vector field, and thus there exists a potential function \(\phi\) such that \( u = \nabla \phi\). Physically, this \(\phi\) is the velocity potential of our fluid. Substituting this into (\ref{mass}), we recover Laplace's equation:

 \begin{equation} \label{laplace}
  \nabla^2 \phi = 0  \\
\end{equation}
 
Adopting the potential formalism, a complete characterization of the trajectories of our system can be reduced to solving for this potential function \(\phi(x,z,t)\). Let us now consider the thin plate equation which forms the basis for our modelling the sheet of ice on the surface of the water, or the elastic vessel wall for a blood vessel. The pressure can be expressed as:

\begin{figure}[hbt!] 
\centering
\includegraphics[scale = .9]{WaterPicture.png}
\caption{ \emph{Waves incident on a thin elastic boundary that deforms with the surface of the fluid. Fluid domain is in the x-z plane and spans \(z=\eta(x,t)\) to \(z=-h\) [\ref{reduced}].}}\label{wavesindeed}
\end{figure}
  

 
\begin{equation} \label{elastic}
  \hspace{-20} D \nabla^4 \eta + \sigma \nabla^2 \eta + \rho_{e} h \frac{\partial \eta^2}{\partial t^2} = p \\
\end{equation}

Where \(\nabla\) is the two dimensional del operator acting in the (x,y) plane, \(\eta\) is the wave profile, \(\rho_e\) is the elastic boundary density, p denotes pressure, \(\sigma\) denotes surface tension, and \(D\) denotes surface flexural-rigidity of the elastic boundary [\ref{bandstructure}]. \(D\) is a structural parameter of the elastic boundary that describes the intrinsic resistance to deformation, or rigidity of the material. It can be expressed as: 
 
\begin{equation} \label{rigidity}
  D = \frac{Ed^3}{12(1-\nu^2)} \\
\end{equation}
 \\
 
 Here \(E\) is the Young's Modulus, which represents the material "stiffness" of our elastic boundary. The parameter d is the thickness of the elastic boundary, and \(\nu\) is the Poisson's ratio, which is a measure of the deformation of a material perpendicular to the direction of loading forces on the elastic boundary. In comparison to \(\ref{elastic}\), the pressure at the interface between the fluid and the elastic boundary can also be expressed as: 

\begin{equation} \label{press}
p = -\rho \frac{\partial \phi}{\partial  t}_{z=0} - \rho g \eta \\
\end{equation}

\vspace{10}

Where \(\rho\) denotes the density of our fluid. 
Let us now consider the boundary conditions of our system. The dynamic boundary condition on the surface of the fluid follows from conservation of momentum, and can be written as follows: 

 \begin{equation} \label{dyn}
  \hspace{100} \phi_{t} + \frac{1}{2} \lvert \nabla \phi \rvert^2 + g \eta = -\frac{p}{\rho}   \hspace{30} (z = \eta(x,t)) \\
\end{equation}

\vspace{10}

The kinematic boundary condition at the surface ensures that a particle on the surface remains on the surface as time evolves in the system. It can be derived in the potential formulation by taking the total derivative of the distance function \(P(x,z,t) = z - \eta(x,t)\). The kinematic boundary condition is:


\begin{equation} \label{kin}
 \hspace{100}\partial_t \eta  + \phi_{x}\eta_{x} = \partial_z \phi \hspace{57}    (z = \eta(x,t))\\
\end{equation}


The final boundary condition to consider is the impermeable bed condition, namely that the z derivative of the velocity potential \(\phi\) is zero at \(z = -h\), which can be considered physically as the bottom of the fluid domain. The bed condition is:

\begin{equation} \label{bed}
 \hspace{100} \partial_z \phi   = 0 \hspace{95}    (z = -h)\\
\end{equation}


We have now defined all the relevant equations in our system. Restricting our analysis to the linear regime and combining (\ref{laplace}), (\ref{elastic}), (\ref{press}), (\ref{dyn}), (\ref{kin}), and (\ref{bed}), we can write our linearized equations of motion as:

 \begin{align} \label{phieq}
    \phi_{t} + g\eta = \frac{1}{\rho}( -D\eta_{4x} - \sigma_{2x} ) \\ \label{etaeq}
    \partial_t \eta  = \partial_z \phi  
 \end{align}
 
Note that these two equations can be combined by taking the time derivative of (\ref{phieq}) and plugging in (\ref{etaeq}), using the commutative property of partial derivatives to eliminate \(\eta\) from the system. Upon doing this, we recover:

\begin{equation} \label{phieq2}
 \phi_{tt} + g\phi_{z} = -\frac{D\phi_{(4x)z}}{\rho} - \frac{\sigma\phi_{(2x)z}}{\rho} \\
\end{equation}
 
\vspace{10} 
 
We will now plug in a small-amplitude solution \(\phi = \epsilon e^{i(kx-\omega t)}\cosh{(k(h+z))} \) into (\ref{phieq}) in order to determine the dispersion relation, and thus the wave speed of our solution. Plugging in this solution, the dispersion relation can be derived: 

\vspace{-20} 

\begin{align}

 \centering{
 -\omega^2 + gk\tanh(kh) = \frac{-Dk^5\tanh(kh)}{\rho} + \frac{\sigma k^3 \tanh(kh)}{\rho}
 
 \\ \vspace{15} \hspace{-500} \implies

 \vspace{15} 
 
 \omega^2 = (gk - \frac{\sigma k^3}{\rho} + \frac{D k^5}{\rho})\tanh{(kh)} 
 }
 \\ 
\end{align}
\vspace{-40}

Taking the square root

\vspace{10}

\begin{equation} \label{dispersion}
 \omega = \sqrt{(gk - \frac{\sigma k^3}{\rho} + \frac{D k^5}{\rho})\tanh{(kh)}}  
 \\
\end{equation}

Using the familiar expression relating wave speed to dispersion \(\omega = ck\):

\begin{equation} \label{V}
 ck = \sqrt{(gk - \frac{\sigma k^3}{\rho} + \frac{D k^5}{\rho})\tanh{(kh)}}
 \\
\end{equation}

Assuming that the solution we are linearizing about is \(2 \pi\) periodic, we can set \(k = 1\) and then solve for the wave speed. Let us define the wave speed, \(V_{0}\) as:\\

\begin{equation} \label{V}
 V_{0} = \sqrt{(g - \frac{\sigma }{\rho} + \frac{D }{\rho})\tanh{(h)}}  
 \\
\end{equation}

It is important to note that physically, either \(D = 0\) and there is no elastic boundary, or \(\sigma = 0\) and there exists an elastic boundary. Both \(\sigma\) and D are kept in our model for completeness, but upon carrying out numerical calculations, at least one will always be set to zero in order for our results to have physical relevance.\\

Varying the surface flexural rigidity D, there should exist modes K of our travelling waves such that  resonance occurs. The resonance condition for a given mode K of these waves \(K \neq 1\) is:

\begin{equation} \label{resonance}
(g + \frac{D}{\rho})K\tanh{(h)} - (g + \frac{DK^4}{\rho})\tanh{(Kh)} = 0 
 \\
\end{equation}

This can also be done with the surface tension \(\sigma\), and the analogous resonance condition is:

\begin{equation} \label{resonancesigma}
(g + \frac{\sigma}{\rho})K\tanh{(h)} - (g + \frac{\sigma K^4}{\rho})\tanh{(Kh)} = 0 
 \\
\end{equation}

\vspace{10} 

We will now employ a change of variables on the surface of the fluid \(q(x,t) = \phi(x,\eta(x,t),t) \) to combine the two equations of motion above into a single, first order equation. It can be shown [\ref{katie}] that upon implementing this change of variables, the two equations reduce to the form:\\

\begin{equation} \label{no}
q_{t} + g\eta = -\frac{1}{\rho}p
\end{equation}

\intertext{\flushleft{which for our problem yields}}

\begin{equation} \label{local}
q_{t} + g\eta = \frac{1}{\rho}( -D\eta_{4x} - \sigma_{2x} )
\end{equation}

\intertext{\flushleft{which we will refer to from now on as the local equation.}}
\vspace{10}

We now approach our stability problem from a slightly different perspective and look to the Ablowitz, Fokas & Musslimani (AFM) formulation [\ref{AFM}] for the non-local behaviour of the system. The non-local equation of motion is derived by rewriting Laplace's equation for the bulk fluid using only surface variables, namely \(\eta\) rather than q [\ref{katie}]. It can be written as:


\begin{equation} \label{no}
\int_{0}^{L} e^{-ikx}(\eta_{t}\cosh(k(h+\eta)) - iq_{x}\sinh(k(h+\eta))) = 0
\end{equation}

Upon considering strictly the linear regime and expanding the hyperbolic sinusoids using the sum trigonometric identities:

\begin{equation} \label{no} 
\cosh(x+y) = \cosh(x)\cosh(y) + \sinh(x)\sinh(y) 
\end{equation}

\begin{equation} \label{no}
\sinh(x+y) = \sinh(x)\cosh(y) + \sinh(y)\cosh(x) 
\end{equation}

\intertext{we recover for our system}

\begin{equation} \label{non-local}
\int_{0}^{L} e^{-ikx}(\eta_{t} - iq_{x}\tanh(kh)) = 0
\end{equation}

\vspace{10}

This equation (\ref{non-local}) will be referred to from now on as the non-local equation in the AFM formulation. \\
\vspace{5}

\subsubsection{The Spectral Stability Problem}

We will now discuss some properties of the spectral stability problem, which we are concerned with in determining the stability regimes of our system. The spectral stability problem in functional analysis is a generalization of the eigenvector-eigenvalue problem to infinite dimensional function spaces. We will use first order perturbation theory in our stability analysis, and so in the context of our system, the eigenfunctions of our spectral stability problem will be the first order perturbations of our velocity potential q and wave profile \(\eta\) at the surface of the fluid. Letting \(w = (q,\eta)\), a general spectral stability problem can be written as:\\


\begin{equation} \label{specstab}
 \mathscr{L}w = \lambda w
\end{equation}

Where \(\mathscr{L}\) is a linear differential operator, \(w\) is a vector of eigenfunctions q and \(\eta\) of the problem, and lambda is an eigenvalue of the problem. Since the problem is infinite dimensional, the set of eigenvalues spans a continuous range of values is thus denoted the \emph{spectrum}. The spectral problem admits a fourier decomposition solution for the eigenfunctions of the problem. \\

\vspace{15} 

Floquet's theorem tells us that all bounded solutions of the spectral problem (12) are of the form: 

\begin{equation} \label{floquet}
  q(x,t) = e^{i\mu x} \sum_{m=-\infty}^{\infty} \hat\mu^{(1)}_m e^{imx} + c.c
\end{equation}

Where \(\mu\) is the period of the perturbation, and m is the mode of a given eigenvalue in the spectrum, and c.c denotes the complex conjugate. Plane waves in the fourier decomposition corresponding with a given fourier mode m have an associated eigenvalue corresponding with the fourier mode m.  Substituting this solution into the spectral problem (\ref{specstab}) and then taking the Fourier transform to leading order in \(\delta_0\) yields an expression for eigenvalues as a function of \(\mu +\)m. Therefore, we have:

\begin{equation} \label{eq1}
  \mathscr{L}q(m+\mu) =   \lambda(m+\mu) q(m+\mu)
\end{equation}

A solution to a PDE is considered spectrally stable if the full spectrum of eigenvalues for the spectral problem lie in the closed left half of the complex plane, that is, if all eigenvalues are either pure imaginary, or have negative real part. This is a generalization of the standard notion of stability of systems of differential equations about equilibria, where negative eigenvalues of the linearized system tend to describe stable sinks, and positive ones describe unstable sources.\\

We seek to classify the stability regimes—that is—the spectral stability of the aforementioned small-amplitude solutions to our equations in various parameter regimes, varying parameters such as the surface flexural rigidity \(D\), the surface tension \(\sigma\), as well as the water depth h. \\



In terms of the scope of the study, we seek to exploit properties of our system that yield necessary conditions for instability of solutions, namely the collisions of eigenvalues of the spectral stability problem, and the aforementioned Krein signature condition of these eigenvalues. As previously mentioned, if all the eigenvalues of the spectral problem have negative or zero real part, then the solutions are stable. However, if eigenvalues have positive real part, then the solutions can be unstable.\\



For our system, if it is the case that there exists eigenvalues with non-zero real part, then that alone is enough to determine the possible instability of the system. This is a result of the necessity that our system is Hamiltonian. Note that under our perturbation, the spectrum of eigenvalues depends continuously on the amplitude of our solution \(q(x,t)\). As the amplitude increases with time, eigenvalues of the spectral problem can diverge from the pure imaginary axis into the real part of the complex plane. In order for this to happen, eigenvalues of different modes on the pure imaginary axis must collide first. If a collision occurs and the eigenvalues spread out into the real part of the complex plane, by the symmetry of the Hamiltonian system, the eigenvalues must come in complex conjugate pairs, as well as pairs reflected across the imaginary axis, so that the Hamiltonian preserves realness, and thus at least one of such eigenvalues would have positive real part, implying instability. This would not be possible if it weren't for the initial collision of the eigenvalues of two different modes. Therefore, another necessary condition for instability is the collision of eigenvalues of the spectral stability problem. Cases of such collisions can be observed in Figure (\ref{eigenvaluesgraph}). \\



\begin{figure}[htp]
\centering
\includegraphics[scale = 0.7]{Eigenvalues.png}
\caption{\emph{Three different configurations of the smallest number of eigenvalues \(\lambda\) of the spectral stability problem of a Hamiltonian system, showing the symmetry about the real and imaginary axes. On the left (in blue), is the stable regime. The centre and right panel are the unstable regimes (in red).}}\label{eigenvaluesgraph}

\end{figure}

\\

The collision condition for two eigenvalues of modes m and n can be written as:


 \begin{equation} \label{eq1}
  \lambda(\mu + m) = \lambda(\mu + n)
\\
\end{equation}

In order to reduce the number of unknowns we are working with, let us set \(\mu \rightarrow \mu + m \) and \(n \rightarrow n-m\). Then, our collision condition reads: 

 \begin{equation} \label{eigenvaluecollisioncondition}
  \lambda(\mu) = \lambda(\mu + n)
\\
\end{equation}

Thus, we are concerned now only with the difference (n-m) in modes of the colliding eigenvalues, rather than the actual values of n and m themselves. Secondly, in order for our Hamiltonian system to conserve energy, for different eigenvalues of modes m and n to collide, they must have opposing signatures so as to preserve the energy, that is if one contributes positively to the energy, the other one has to contribute negatively. Thus, we need a condition that lets us know that our eigenvalues of modes m and n are of opposing signature. Such a condition is called the Krein signature condition:

 \begin{equation} \label{krein1}
  s = (\mu + n)(\mu + m) < 0
\\
\end{equation}

Again, making the substitutions \(\mu \rightarrow \mu + m \) and \(n \rightarrow n-m\), we recover an analagous expression to (16) for our signature of colliding eigenvalues condition:

 \begin{equation} \label{krein2}
  s = (\mu)(\mu + n) < 0
\\
\end{equation}


Using these conditions (\ref{eigenvaluecollisioncondition}) and (\ref{krein2}), as well as the expression for the eigenvalues \(\lambda\), a stability function can be defined, the roots of which determine the spectral stability of the system. If the roots are purely imaginary, then the solutions are stable, but if the roots are real, then the system can have instabilities.  We seek to examine the conditions on the coefficients of the aforementioned stability function that will necessitate purely imaginary eigenvalues, and thus the spectral stability of the wave solutions. \\

\subsubsection{The Resonance Method of Ice Destruction}

Lastly, we will consider the resonant regimes of the water-wave ice system. A common procedure in the applied physical sciences in the field of hydrodynamics is called the resonance method of ice destruction. The theory behind the method is that when the magnitude of the phase velocity of a heavy load moving on the ice is close to the magnitude of the phase velocity of the flexural-gravity waves underneath, the water no longer supports the ice against gravity, the system is in resonance, and the amplitudes of the flexural-gravity waves grow exponentially such that eventually the ice breaks. An application of this method is hovercraft 'icebreaking', used in order to clear passageways for water transportation, as demonstrated in Figure (\ref{Hovercraft}) below. 
\\

\begin{figure}[hbt!]
\centering
\includegraphics[scale = .4]{Hovercraft.jpg}
\caption{ \emph{A hovercraft moving on ice and demonstrating the resonance method of ice destruction. The hovercraft matches the linear phase velocity of the flexural-gravity waves under the ice and induces resonance-based amplitude spikes that break the ice as they propagate.}} \label{Hovercraft}
\end{figure}



\subsection{Literature Review}

Stability regimes have been almost completely determined for certain classes of water wave solutions, such as the standard soliton solutions of the Korteweg-de-Vries (KdV) equation [2], and the solitary wave solutions of the Boussinesq equation [3]. For surface flexural-gravity waves, stability regimes have yet to be fully characterized under the full variation of physically relevant parameters. Trichtchenko, Milewski, Parau, and Vanden-Broeck analyzed the stability of periodic two dimensional travelling flexural-gravity waves for the deep water regime \(h = \infty\), and through solving the non-linear Schrodinger equation derived from Euler’s equations, it was found that in the regime of large surface flexural-rigidity \((D \sim 25)\), non-linear models tend to exhibit stable solutions, deviating from the behaviour of their linear counterparts, which exhibit high-frequency instabilities in this regime [4]. Gao, Wang, and Vanden-Broeck also studied the dynamics of propagating deep water flexural-gravity waves under a floating flexible plate (ice) [5]. The surface ice exerts pressure on the water, which accounts for the non-linearity of the model. A numerical simulation of the wave solutions was computed, encapsulating the response to a localized pressure distribution moving at a rate slightly under the wave phase speed—this is known as the trans-critical regime. Symmetric, moderate-amplitude solitary wave solutions to Euler’s irrotational water wave equations with flexural-gravity boundary conditions were numerically found to have finite amplitude within this regime, and thus were determined as stable. 
\\



There is also extensive research investigating the stability of shallow water waves. Shallow water waves of large wavelength are denoted cnoidal waves, whereas waves of shorter wavelength are denoted Stokes waves. The stability of general periodic, small amplitude, shallow water waves has been found to depend on the difference in wavelengths between the perturbation and the fundamental wave solution (zero-th order perturbative term), as well as the nonlinearity of the perturbation. [6]. Small amplitude perturbations with wavelengths either equal to or slightly greater than the wavelength of the fundamental wave solution were found to be stable. Upon varying the perturbation wavelength, it was determined that as the wavelength of the perturbation becomes large compared to the fundamental wavelength, the stability of the wave solutions decreases. Hammack et. al studied the stability of two dimensional periodic waves in shallow water, using experimental data that demonstrated the existence and stability of such propagating waves, as well as the accuracy of fitting the family of exact solutions of the Kadomtsev-Petviashvili (KP) equation to the data [7]. Such solutions can be thought of as two dimensional generalizations of cnoidal waves, and were found to be stable even under large perturbations. 
\\



Presently, much research has been performed attempting to characterize the stability regimes of deep water, surface flexural-gravity waves. However, the effect of variation in parameters such as the surface flexural-rigidity, and surface tension of these systems in the shallow water regime \(h \leq 1.33\) has not been thoroughly studied, and thus will be a focus of this paper. Stability properties of general classes of shallow water wave solutions have been determined, but not in great detail for the aforementioned hydro-elastic waves in question.  In addition, little research has been conducted investigating conditions on the spectral stability function that affect stability of wave solutions. We intend to formulate a set of necessary criteria for such functions to determine resultant wave stability. The advantages of studying shallow-water surface flexural-gravity waves lie in the scarcity of literature attempting to characterize stability regimes of such systems, as well as possible environmental applications and extensions of the research. Disadvantages of the study include the lack of pre-existing efforts to quantify the aforementioned stability regimes in questions—this would help as a comparative benchmark for our findings, as well as the highly non-linear dispersion relations for such equations, making our spectral stability functions difficult to compute due to the presence of hyperbolic tangents, and fifth order surface-flexural terms. We intend to mitigate the latter disadvantage by solving an eigenvalue matrix constructed from the local and non-local equations (\ref{local}) and (\ref{non-local}) in order to determine an analytical expression for the eigenvalue spectrum. The aforementioned considerations from the literature will be kept in mind to help inform the progress and completion of this study. \\

We hypothesize that our solutions to shallow-water, flexural-gravity water wave equations will exhibit stability in multiple regimes of physically appropriate values of the surface flexural rigidity, surface water tension, and water depth. Concerning the eigenvalues of the spectral problem, for colliding eigenvalues of different modes m and n, there should exist values for the difference between m and n such that the system can exhibit instabilities, and specific parameter regimes such that these instabilities can occur. 
\\

\section{Methodology}

The introduction and literature review will now be followed by a detailed description of our experimental methodology, followed by our results and then a conclusion and discussion of the findings. 
 \\
 
\subsection{First Order Perturbations of The Non-Local Equation}

For our stability analysis for the water wave problem, we start from the equations which are, respectively:

\begin{align} \label{system}
q_{t} + g\eta = \frac{1}{\rho}( -D\eta_{4x} - \sigma_{2x} )\\\label{system2}
\int_{0}^{L} e^{-ikx}(\eta_{t} - iq_{x}\tanh(kh)) = 0
\end{align}
\\

Let us start with the non-local equation (\ref{system2}). We now implement first-order perturbation theory on our solutions q and \(\eta\). The zero'th order perturbation terms drop out of our system and we are left with the first order terms, which can be expressed via Floquet theory (\ref{floquet}). The first-order perturbations for q and \(\eta\) can be written as

\begin{equation} \label{q}
  q^{(1)}(x,t) = \delta_{0}e^{\lambda t}\sum_{m=-\infty}^{\infty}b_{m}e^{i(m+\mu)(x-V_{0}t)}
\\
\end{equation}


\begin{equation} \label{eta}
  \eta^{(1)}(x,t) = \delta_{0}e^{\lambda t}\sum_{m=-\infty}^{\infty}a_{m}e^{i(m+\mu)(x-V_{0}t)}
\\
\end{equation}

We now seek to determine an analytical result for our spectral stability eigenvalue function by setting up a 2x2 matrix for this system and solving for the eigenvalues \(\lambda\). We shall now derive the expression that will constitute the first row of this matrix. Plugging (\ref{q}) and (\ref{eta}) into the non-local equation (\ref{system2}), we recover the expression:

\begin{align}

    \int_{0}^{L} e^{-ikx}((\lambda - i(m+\mu)V_{0}\delta_{0}e^{\lambda t}\sum_{m=0}^{\infty}a_{m}e^{i(m+\mu)(x-V_{0}t)} \hspace{2} - \hspace{3}  i(i(m+\mu)\delta_{0}e^{\lambda t}\sum_{m=0}^{\infty}b_{m}e^{i(m+\mu)(x-V_{0}t)}\tanh(kh))dx = 0
    
    \\ \vspace{15} \implies

    \vspace{15}
    \int_{0}^{L} e^{-ikx}((\lambda - i(m+\mu)V_{0}\delta_{0}e^{\lambda t}\sum_{m=0}^{\infty}a_{m}e^{i(m+\mu)(x-V_{0}t)} \hspace{2} + (m+\mu)\delta_{0}e^{\lambda t}\tanh(kh)\sum_{m=0}^{\infty}b_{m}e^{i(m+\mu)(x-V_{0}t)}dx = 0
    
    \\ \vspace{15} \implies

    \vspace{15}
    \centering
    \int_{0}^{L} e^{\lambda - iV_{0}t}\sum_{m=0}^{\infty}[(\lambda - i(m+\mu)V_{0})\delta_{0}a_{m} + \tanh(kh)(m+\mu)\delta_{0}b_{m} ] e^{i(m+\mu-k)x} = 0     
    
    \\ \vspace{15} \flushleft{assuming well-behavedness of our perturbative solutions, we can interchange the sum and integral and evaluate the plane wave integral as follows:}
    
    \vspace{15}
    \centering{
    e^{\lambda - iV_{0}t}\sum_{m=0}^{\infty}[(\lambda - i(m+\mu)V_{0})\delta_{0}a_{m} + \tanh(kh)(m+\mu)\delta_{0}b_{m} ] \frac{e^{i(m+\mu-k)x}}{i(m+\mu-k)}\Big|_0^L + C = 0 }    
    \\ \vspace{15} 
    
    \flushleft{without loss of generality we can set C = 0 and then evaluate this expression at the upper/lower limits of the integral:}
    
    \vspace{15}
    \centering{
    e^{\lambda - iV_{0}t}\sum_{m=0}^{\infty}[(\lambda - i(m+\mu)V_{0})\delta_{0}a_{m} + \tanh(kh)(m+\mu)\delta_{0}b_{m} ] \frac{e^{i(m+\mu-k)L} - 1}{i(m+\mu-k)} = 0 }  
    
    \\ \vspace{15} \flushleft{In attempts to simplify this expression slightly, we factor \(e^{i(m+\mu+k)\frac{L}{2}}\) out of the expression:}
    
    \vspace{15}
    \centering{
    e^{\lambda - iV_{0}t}\sum_{m=0}^{\infty}[(\lambda - i(m+\mu)V_{0})\delta_{0}a_{m} + \tanh(kh)(m+\mu)\delta_{0}b_{m} ]e^{i(m+\mu-k)\frac{L}{2}}[\frac{e^{i(m+\mu-k)\frac{L}{2}} - e^{-i(m+\mu-k)\frac{L}{2}}}{i(m+\mu-k)}] = 0 }  
    
    \\ \vspace{15} \flushleft{recognizing that  \(\sin((m + \mu + k)\frac{L}{2}) = \frac{e^{i(m+\mu-k)\frac{L}{2}} - e^{-i(m+\mu-k)\frac{L}{2}}}{2i}\), we can rewrite the above expression as:}
    
    \vspace{15}
    \centering{
    e^{\lambda - iV_{0}t}\sum_{m=0}^{\infty}[(\lambda - i(m+\mu)V_{0})\delta_{0}a_{m} + \tanh(kh)(m+\mu)\delta_{0}b_{m} ]e^{i(m+\mu-k)\frac{L}{2}}[\frac{2\sin((m + \mu - k)\frac{L}{2})}{(m+\mu-k)}] = 0 }  
    
    \vspace{15} \flushleft{Using the orthogonality of the plane wave basis, we can recognize that under an inner product with the basis function from the n'th fourier mode:}
    
    \vspace{15}
    \centering{
    \int_{0}^{L} e^{\lambda - iV_{0}t}\sum_{m=0}^{\infty}[(\lambda - i(m+\mu)V_{0})\delta_{0}a_{m} + \tanh(kh)(m+\mu)\delta_{0}b_{m} ]\frac{2\sin((m + \mu - k)\frac{L}{2})}{(m+\mu-k)}[e^{i(m+\mu-k)\frac{L}{2}}e^{i(n+\mu-k)\frac{L}{2}}]dx = 0 }  
    
    \\ \vspace{15} \flushleft{The inner product collapses the sum in this expression since all terms vanish other than the case where \(n = m\)}
    
    \vspace{15}
    \centering{
    \int_{0}^{L} e^{\lambda - iV_{0}t}[(\lambda - i(m+\mu)V_{0})\delta_{0}a_{m} + \tanh(kh)(m+\mu)\delta_{0}b_{m} ]\frac{2\sin((m + \mu - k)\frac{L}{2})}{(m+\mu-k)}[e^{i(m+\mu-k)L}]dx = 0 } 
    
    \\ \vspace{15} \flushleft{Upon evaluation of this integral, we recover:}
    
    \vspace{15}
    \centering{
    e^{\lambda - iV_{0}t}[(\lambda - i(m+\mu)V_{0})\delta_{0}a_{m} + \tanh(kh)(m+\mu)\delta_{0}b_{m} ]\frac{2\sin((m + \mu - k)\frac{L}{2})}{(m+\mu-k)}e^{i(m+\mu-k)L}L = 0 } 
\end{align}

\vspace{-20}

Since this equation is equal to 0 we can divide constant terms out on the left-hand-side to obtain the first row of the matrix:

\vspace{15} 

\begin{equation} \label{matrixrow1}
   [(\lambda - i(m+\mu)V_{0})\delta_{0}a_{m} + \tanh(kh)(m+\mu)\delta_{0}b_{m} ] = 0 
\\
\end{equation}

\vspace{15}

\subsection{First Order Perturbations of The Local Equation}

\vspace{15}

Now, we look to the local equation (\ref{system}) and we plug in the first order perturbations (\ref{q}) and (\ref{eta}). In doing so, we derive the second row of the stability eigenvalue matrix we seek.

\vspace{-10} 

\begin{align}

 \centering{
 \sum_{m = 0}^{\infty} \bigg[(g\delta_{0} + \frac{D\delta_{0}(m+\mu)^4}{\rho} - \frac{\sigma \delta_{0} (m+\mu)^2}{\rho})a_{m} - (\lambda - i(m+\mu)V_{0})\delta_{0}b_{m}\bigg]e^{i(m+\mu-k)x} = 0 
 }
    
 \vspace{15} \flushleft{{Using the orthogonality of the plane wave basis, we can recognize that under an inner product with the basis function from the n'th fourier mode:}}
 
 \vspace{15}
 \centering{
 \int_{0}^{L} \sum_{m = 0}^{\infty} \bigg[(g\delta_{0} + \frac{D\delta_{0}(m+\mu)^4}{\rho} - \frac{\sigma \delta_{0} (m+\mu)^2}{\rho})a_{m} - (\lambda - i(m+\mu)V_{0})\delta_{0}b_{m}\bigg]e^{i(m+\mu-k)x}e^{i(n+\mu-k)x} = 0 }

\vspace{15} \flushleft{Again, the inner product collapses the sum in this expression since all terms vanish other than the case where \(n = m\)}
 
 \vspace{15}
 
 \centering{
 \int_{0}^{L}\bigg[(g\delta_{0} + \frac{D\delta_{0}(m+\mu)^4}{\rho} - \frac{\sigma \delta_{0} (m+\mu)^2}{\rho})a_{m} - (\lambda - i(m+\mu)V_{0})\delta_{0}b_{m}\bigg]e^{2i(m+\mu-k)x} = 0 
 }
 
 \vspace{15} \flushleft{Evaluating this integral and plugging in the bounds, we get:}
 
 \vspace{15}
 
 \centering{
 \bigg[(g\delta_{0} + \frac{D\delta_{0}(m+\mu)^4}{\rho} - \frac{\sigma \delta_{0} (m+\mu)^2}{\rho})a_{m} - (\lambda - i(m+\mu)V_{0})\delta_{0}b_{m}\bigg]\frac{e^{2i(m+\mu-k)x}}{2i(m+\mu-k)}\Big|_{0}^{L} = 0 }
 
 \vspace{15} \flushleft{Recognizing that the expression with the bounds on the right is just a constant, we can divide through to obtain:}
  
 \vspace{-30} 
  
 \end{align}
 
\begin{equation} \label{matrixrow2}
 \centering{
  \bigg[(g\delta_{0} + \frac{D\delta_{0}(m+\mu)^4}{\rho} - \frac{\sigma \delta_{0} (m+\mu)^2}{\rho})a_{m} - (\lambda - i(m+\mu)V_{0})\delta_{0}b_{m}\bigg] = 0 }
\end{equation}

\vspace{15} 

which is simply the second row of our stability matrix.

\subsection{Constructing and Solving the Stability Eigenvalue Matrix}

Now, we we combine the first and second rows to construct the stability matrix. Using (\ref{matrixrow1}) and (\ref{matrixrow2}), we can write: 


\begin{align} \label{matrix}
  \bigg[(g\delta_{0} + \frac{D\delta_{0}(m+\mu)^4}{\rho} - \frac{\sigma \delta_{0} (m+\mu)^2}{\rho})a_{m} - (\lambda - i(m+\mu)V_{0})\delta_{0}b_{m}\bigg] = 0 
  \\
  [(\lambda - i(m+\mu)V_{0})\delta_{0}a_{m} + \tanh(kh)(m+\mu)\delta_{0}b_{m} ] = 0 
\end{align}

rearranging to put the \( \lambda \) terms on the left side, we obtain the system of equations:



\begin{align} \label{matrix2}
  \lambda a_{m} =  (i(m+\mu)V_{0})a_{m} - (\tanh(kh)(m+\mu))b_{m} 
  \\
  \lambda b_{m} =  (g - \frac{\sigma(m+\mu)^2}{\rho} + \frac{D(m+\mu)^4}{\rho})a_m + (i(m+\mu)V_{0})b_{m} 
\end{align}

This system of equations can be cast into matrix form as: 

\begin{equation}
    \begin{bmatrix} \label{stabilitymatrix}
    \lambda & 0\\
    0 & \lambda 
    \end{bmatrix}
    = 
    \begin{bmatrix} \label{stabilitymatrix}
    i(m+\mu)V_{0} & -\tanh(kh)(m+\mu)\\
    g - \frac{\sigma(m+\mu)^2}{\rho} + \frac{D(m+\mu)^4}{\rho} & i(m+\mu)V_{0} 
    \end{bmatrix}
\end{equation}

\vspace{10} 

Upon solving the eigenvalue problem for this matrix, we recover an analytical expression for the eigenvalue spectrum as a function of \(\mu+n\) 


\begin{equation} \label{eigenvalue}
  \lambda(m + \mu) = \bigg[\pm\sqrt{(g(m+\mu) - \frac{\sigma(m+\mu)^3}{\rho} + \frac{D(m+\mu)^5}{\rho})\tanh{((m+\mu)h)}}\bigg]i + V_{0}(m+\mu)i 
\\
\end{equation}


\vspace{15}

\subsection{The Four Spectral Stability Functions}

As a reminder, the eigenvalue collision condition for potential instability is:


\begin{equation} \label{eigenvaluecollisioncondition2}
  \lambda(\mu) = \lambda(\mu + n)
\\
\end{equation}

and the Krein signature of colliding eigenvalues condition is:

 \begin{equation} \label{krein2b}
  s = \mu(\mu + n) < 0
\\
\end{equation}


Rearranging (\ref{eigenvaluecollisioncondition2}) such that the right hand side is equal to 0, we can define the aforementioned stability function in question. However, since the eigenvalue spectrum function \(\lambda\) can take either a positive or negative sign infront of the square root, it follows that there actually exists four different stability functions, each with different combinations of 'signs' of \(\Omega(\mu + n)\) and \(\Omega(\mu)\). The stability functions p are defined as:

 \begin{equation} \label{eq1}
  p(\mu,n) = \Omega(\mu + n) - \Omega(\mu)
\\
\end{equation}

where \(\Omega(x) = i\lambda(x)\). The four stability functions can be written out as: 

 \begin{equation} \label{p1}
  \small{p_1(\mu,n) = \sqrt{(g(n+\mu) - \frac{\sigma(n+\mu)^3}{\rho} + \frac{D(n+\mu)^5}{\rho})\tanh{((n+\mu)h)}}   + \sqrt{(g(\mu) - \frac{\sigma(\mu)^3}{\rho} + \frac{D(\mu)^5}{\rho})\tanh{((\mu)h)}} - V_{0}n}
\\
\end{equation}

\begin{equation} \label{p2}
  \small{p_2(\mu,n) = \sqrt{(g(n+\mu) - \frac{\sigma(n+\mu)^3}{\rho} + \frac{D(n+\mu)^5}{\rho})\tanh{((n+\mu)h)}}   - \sqrt{(g(\mu) - \frac{\sigma(\mu)^3}{\rho} + \frac{D(\mu)^5}{\rho})\tanh{((\mu)h)}} - V_{0}n}
\\
\end{equation}

\begin{equation} \label{p3}
  \small{p_3(\mu,n) = - \sqrt{(g(n+\mu) - \frac{\sigma(n+\mu)^3}{\rho} + \frac{D(n+\mu)^5}{\rho})\tanh{((n+\mu)h)}}   + \sqrt{(g(\mu) - \frac{\sigma(\mu)^3}{\rho} + \frac{D(\mu)^5}{\rho})\tanh{((\mu)h)}} - V_{0}n}
\\
\end{equation}

\begin{equation} \label{p4}
  \small{p_4(\mu,n) = - \sqrt{(g(n+\mu) - \frac{\sigma(n+\mu)^3}{\rho} + \frac{D(n+\mu)^5}{\rho})\tanh{((n+\mu)h)}}   - \sqrt{(g(\mu) - \frac{\sigma(\mu)^3}{\rho} + \frac{D(\mu)^5}{\rho})\tanh{((\mu)h)}} - V_{0}n}
\\
\end{equation}

These functions are the result of the instability conditions (\ref{eigenvaluecollisioncondition2}) and (\ref{krein2b}), as well as the dispersion relation of our equations of motion. Consider the Krein instability condition (\ref{krein2b}). This condition can be rearranged to yield the quadratic:

 \begin{equation} \label{mupoly}
   \mu^2 + n \mu - s = 0 
\\
\end{equation}

with \(s<0\) specified by (\ref{krein2b}). The solutions for this quadratic yield roots \(\mu\) which can be considered as functions of both n and s. By the quadratic formula:

 \begin{equation} \label{muroots}
   \mu(n,s) = \frac{-n \pm \sqrt(n^2 - 4s)}{2}
\\
\end{equation}

We seek to determine the realness of the roots of (\ref{mupoly}) for fixed values of n, but on what interval? Both instability conditions (\ref{eigenvaluecollisioncondition2}) and (\ref{krein2b}) must be satisfied. Thus, the roots (\ref{muroots}) of the quadratic (\ref{mupoly}) must be real. Requiring that the discriminant be positive, it follows that the sufficient condition for realness of (\ref{muroots}) is:

 \begin{equation} \label{interval}
  -\frac{n^2}{4} < s < 0
\\
\end{equation}

Let us now substitute the Krein condition (\ref{krein2b}) into s in (\ref{interval}) to get: 

 \begin{equation} \label{eqn3}
  -\frac{n^2}{4} < \mu(\mu + n) < 0
\\
\end{equation}
\intertext{upon completing the square, we find}
 \begin{align} \label{eqn3}
  -\frac{n^2}{4} < (\mu + \frac{n}{2})^2 - \frac{n^2}{4} < 0
\\
 \implies 0 < (\mu + \frac{n}{2})^2 < \frac{n^2}{4}
\\
 \hspace{10} \implies -\frac{n}{2} < (\mu + \frac{n}{2}) < \frac{n}{2} 
\\
\implies -\frac{n}{2} < (\mu + \frac{n}{2}) < \frac{n}{2} 
\end{align}
\intertext{which after subtracting \(\frac{n}{2}\) from all three sides, can be reduced to the inequality}

 \begin{equation} \label{interval2}
  -n < \mu < 0
\\
\end{equation}

Thus, since this inequality (\ref{interval}) above must be satisfied for instability to occur, and this inequality is equivalent to (\ref{interval2}),
we must show that (\ref{p1}), (\ref{p2}), (\ref{p3}), and (\ref{p4}) all have no real roots in the interval (\ref{interval2}) for our solution to be determined stable. If there exists even one root in this interval, then that means that the instability conditions (\ref{eigenvaluecollisioncondition2}) and (\ref{krein2b}) are satisfied for this value of \(\mu\), and thus instability can occur. In order to test for real roots of an arbitrary continuous function on an interval of the real line, we can use the secant method for root finding. \\

The secant method was implemented in python and is defined recursively by the sequence: 

 \begin{equation} \label{secant}
  x_{n} = \frac{ x_{n-2}f(x_{n-1}) - x_{n-1}f(x_{n-2}) }{f(x_{n-1}) - f(x_{n-2})}
\\
\end{equation}
\intertext{where \(x_{0}\) and \(x_{1}\) are defined by the left and right points of the interval for a given value of n (\ref{interval2})}.

\vspace{10}
Thus, we have our method for determining the realness of roots to the polynomial (\ref{mupoly}), and thus the stability of our solutions to our equation of motion for fixed values of parameters D, \(\sigma\) and h. The next step is to vary these aforementioned parameters and investigate the effects of such changes on the existence of real roots in the interval (\ref{interval2}), and thus the stability of the system. \\

\subsection{Resonant Regimes of The Water-Wave System}

We will also examine resonant regimes of the system, and see how they interact with the high-frequency instabilities under study. We previously described how the resonance method of ice destruction exploits the fact that when a heavy load on the ice is moving at a velocity close to the phase speed of the flexural-gravity waves underneath, resonant destruction of ice occurs. Consider our perturbative first order solution (\ref{q}) given by Floquet theory. Let us for simplicity's sake define \(k_{p} = m + \mu\). Then (\ref{q}) becomes:

\begin{equation} \label{q2}
  q^{(1)}(x,t) = \delta_{0}e^{\lambda t}\sum_{m=-\infty}^{\infty}b_{m}e^{i(k_{p})(x-V_{0}t)}
\\
\end{equation} 

A 1D wave equation for the perturbation is:

\begin{equation} \label{qwave}
  \frac{\partial^2{q^{(1)}}}{\partial{t}^2} = c_{p}^2 \frac{\partial^2{q^{(1)}}}{\partial{x}^2}
\\
\end{equation} 

Our perturbation can model a moving load on the ice, and thus we can find the load velocity on the ice by a method analogous to how we determined the dispersion relation (\ref{dispersion}). Noting that the perturbation is in essence, a fourier decomposition of some scalar function that solves the spectral problem; this scalar function is a perturbation of the velocity potential that can represent the contribution of the dynamics of a moving load on the ice. Plugging (\ref{q2}) into (\ref{qwave}) and using the spectrum equation (\ref{eigenvalue}), we recover the expression for the linear perturbation velocity \(c_{p}\): 

\begin{equation} \label{cp}
  c_{p} = \pm \frac{\sqrt{(g(k_{p}) - \frac{\sigma(k_{p})^3}{\rho} + \frac{D(k_{p})^5}{\rho})\tanh{(k_{p}h)}}}{k_{p}}
\\
\end{equation} 

Thus, a condition equivalent to the resonance condition stated previously (\ref{resonance}) is the condition that the magnitude of the phase velocity of the flexural gravity waves \(V_{0}\) is equivalent to that of the linear perturbation velocity (\ref{cp}), that is:

\begin{equation} \label{resonance2}
  |c_{p}| = |V_0| \\
\end{equation} 


\\

\subsection{Summary}

The methodology for our spectral stability analysis can be summarized via the implementation of six simple steps generalizing the stability analysis of Trichtchenko to this higher dimensional system [\ref{reduced}].

\begin{enumerate}
  \item Write down the local (\ref{local}) and non-local (\ref{non-local}) equations representative of the water wave problem emergent from Euler's equations and the AFM formulation, respectively.
  \item Write the solutions to q and \(\eta\) using first order perturbation theory via Floquet's theorem (\ref{floquet}). Then write these solutions in the travelling frame of reference, making the substitution (x \implies x - \(V_{0}\)t)
  \item Plug in the first order perturbations into the non-local and local equations to derive a 2x2 stability matrix, the eigenvalues of which represent the spectrum of the spectral stability problem. Use the orthonormality property of the basis functions in the Fourier-Floquet decomposition and the inner product to pick out the arbitrary m'th terms in the series for the matrix, and then solve the eigenvalue eigenvector problem for an eigenvalue expression as a function of \(\mu + n\)
  \item From the eigenvalue collision condition (\ref{eigenvaluecollisioncondition2}), write the stability functions \(p(\mu,n) = \Omega(\mu + n) - \Omega(\mu)\) where \( \Omega = \pm i\lambda\). 
  \item Use the secant method for root finding to determine if there exists any roots in the interval prescribed by the eigenvalue collision condition (\ref{eigenvaluecollisioncondition2}) and condition of opposing eigenvalue signatures (\ref{krein2b}). 
  \item Vary the parameters D, \(\sigma\) and h to see the effects of different physical parameter values of the flexural rigidity, surface tension, and water depth on the stability of the system. Consider the effects of resonance on the system.
\end{enumerate}
\\


\begin{table}[hbt!]
    \begin{center}
        \begin{tabular}{ |c|c|c|c| } 
        \hline
        Parameters & Range (Water-Ice) & Range (Blood Vessel) & SI Units \\
        \hline
        D & 5.61 x 10^8 - 1.12 \times 10^9 & TBA & N\cdot m^2\\
        \hline
        \(\sigma\) & 0 - 7 & TBA & N\cdot m  \\
        \hline
        h & 0 - 10 & TBA & m \\
        \hline
        d & 0 - 10 & TBA & m \\
        \hline
        \(\rho\) & 1000 & TBA & \(\frac{kg}{m^3}\) \\
        \hline
        g & 9.8 & TBA & \(\frac{m}{s^2}\) \\
        \hline
        E & 8.7 \(\times\) 10^9 & TBA & \frac{N}{m^2}  \\
        \hline
        \(\nu\) & 0.33 & 0.17 & N/A \\ 
        \hline
        \end{tabular} 
        \caption{\label{tab:parameters}\emph{Physical parameter ranges for our water wave system. Note that for completeness we include \(\sigma\) and D together in our equations, but physically if there exists an elastic boundary in the system, then we have flexural-gravity waves where \(\sigma\) is necessarily 0, while D is non-zero, and then the converse is true for when there does not exist an elastic boundary in the system, and we have capillary-gravity waves.}}
    \end{center}
\end{table}
\\

\clearpage 
\section{Results}

\subsection{Varying Constant Parameters of the System (D, \(\sigma\), h, 'n')}

Plots of the spectral stability functions (\ref{p1}), (\ref{p2}), (\ref{p3}), and (\ref{p4}) for varying values of the parameters D, n, h, and \(\sigma\) are given below. \\

Consider the interval (\ref{interval}) that we seek to check for real roots in for a fixed value of n. Varying the Fourier mode difference n parameter has the effect of expanding the interval along the -x axis. The effects of this can be visualized below, here we have fixed the thickness of our elastic boundary to 2 centimeters, and \(h = 0.5\). The lack of existence of roots, and thus spectral stability is found to vary with different values of n, as demonstrated below. \\


\begin{figure}[hbt!]
\centering
\includegraphics[width=.3\textwidth]{n0.png}\hfill
\includegraphics[width=.3\textwidth]{n1.png}\hfill
\includegraphics[width=.3\textwidth]{n2.png}\hfill
\includegraphics[width=.3\textwidth]{n3.png}\hfill
\includegraphics[width=.3\textwidth]{n4.png}\hfill
\includegraphics[width=.3\textwidth]{n5.png}\hfill
\caption{\emph{Spectral Stability Functions for n from 0 to 5. \(D = 0.02\) m, \(\sigma = 0\), and \(h = 0.5\)}} \label{nvary}
\end{figure}



Generally, as D gets larger, the effect on the stability functions is horizontal compression which can lead to new roots appearing in our interval (\ref{interval}), The effects of variation of the surface flexural rigidity on the stability functions can be visualized in Figure \((\ref{Dvary})\). Note the qualitative similarity of the behaviour of the spectral stability functions to Figure (\ref{nvary}) as we now increase the D parameter, rather than the n parameter. \\

\begin{figure}[hbt!]
\centering
\includegraphics[width=.3\textwidth]{D0.png}\hfill
\includegraphics[width=.3\textwidth]{D1.png}\hfill
\includegraphics[width=.3\textwidth]{D2.png}\hfill
\includegraphics[width=.3\textwidth]{D2real.png}\hfill
\includegraphics[width=.3\textwidth]{D3.png}\hfill
\includegraphics[width=.3\textwidth]{D4.png}\hfill
\caption{\emph{Spectral Stability Functions for D from 0 to \(10^4\). \(n = 5\), \(\sigma = 0\), and \(h = 0.5\)}} \label{Dvary}
\end{figure}

\\



It can be seen that varying the surface tension \(\sigma\) throughout the physical regimes produces little to no discernible effect on the root behaviour of the spectral stability functions. This can be observed in Figure (\ref{sigmavary}) below. \\



\begin{figure}[hbt!]
\centering
\includegraphics[width=.3\textwidth]{sigma1.png}\hfill
\includegraphics[width=.3\textwidth]{sigma2.png}\hfill
\includegraphics[width=.3\textwidth]{sigma3.png}\hfill
\includegraphics[width=.3\textwidth]{sigma4.png}\hfill
\includegraphics[width=.3\textwidth]{sigma5.png}\hfill
\includegraphics[width=.3\textwidth]{sigma6.png}\hfill
\caption{\emph{Spectral Stability Functions for \(\sigma\) from 0 to 10. \(n = 5\), \(D = 0\), and \(h = 0.5\)}} \label{sigmavary}
\end{figure}


\\
As the water depth parameter h goes to zero, the spectral stability functions approach the line \(y = 0\) as well. In the shallow water regime for our \(2\pi\) periodic waves, \((h \leq 1.33)\), the stability functions can exhibit root behaviour depending on the values of D, \(\sigma\), and n. Fixing these values, in the deep water regime, \((h \leq 1.33)\), the spectral stability functions exhibit increased curvature that does not seem to introduce or eliminate roots in the interval (\ref{interval}) in comparison with the shallow water regime. These results can be seen in Figure (\ref{hvary}). \\

\clearpage

\begin{figure}[hbt!]
\centering
\includegraphics[width=.3\textwidth]{h1.png}\hfill
\includegraphics[width=.3\textwidth]{h2.png}\hfill
\includegraphics[width=.3\textwidth]{h3.png}\hfill
\includegraphics[width=.3\textwidth]{h4.png}\hfill
\includegraphics[width=.3\textwidth]{h5.png}\hfill
\includegraphics[width=.3\textwidth]{h6.png}\hfill
\caption{\emph{Spectral Stability Functions for h from 0 to 1000. Notice the three bottom graphs represent deep water \(h >= 1.33)\) while the three top graphs represent the shallow water regime \(h < 1.33)\). Here \(n = 5\), \(\D = 100\), and \(\sigma = 0\)}} \label{hvary}
\end{figure}



\\

\subsection{Varying Elastic Boundary Thickness d and Fourier Mode Difference Parameter  n Simultaneously Yields Various Stability Regimes}

When performing the spectral stability analysis in Python, stability regimes were investigated upon varying the flexural rigidity \(D\) and fourier mode difference parameter n. Since the range of \(D\) varies drastically on the order of \((1-10^9)\), instead of directly varying \(D\), we used our elasticity theory equation for surface-flexural rigidity (\ref{rigidity}) and varied the thickness of the elastic boundary. The flexural rigidity is proportional to the thickness cubed, and thus evaluating stability regimes varying the thickness d with Fourier mode difference n offers a more instructive and clear comparison. \\

Stability outcomes were computed in Python for two different thickness regimes. In the millimeter regime, existence of real roots was checked numerically for approximately 2500 'coordinate' values of \((n,d)\), incrementally varying the parameters n and d at evenly spaced intervals. The same process was performed for the centimeter regime, but this time with \(10^4\) coordinate values, in order to encapsulate finer details of the stability boundaries in the regime. The figures below describe the stability outcomes for all coordinates for each regime, with red corresponding to unstable coordinates \((n,d)\) and blue corresponding to stable coordinates \((n,d)\). \\

\begin{figure}[hbt!]
\centering
\includegraphics[scale = .8]{Stability CM .png}
\caption{ \emph{Stability regimes varying ice thickness in the centimeter regime with Fourier mode difference parameter n. Stable regions are denoted in "blue", whereas unstable regions are denoted in "red". Here, \(h = 0.5\), \(\sigma = 0\). }} \label{CM}
\end{figure}


\begin{figure}[hbt!]
\centering
\includegraphics[scale = .8]{Stability MM.png}
\caption{\emph{Stability regimes varying blood vessel thickness in the millimeter regime with Fourier mode difference parameter n. Stable regions are denoted in "blue", whereas unstable regions are denoted in "red". Here, \(h = 0.5\), \(\sigma = 0\)}} \label{MM}
\end{figure}

\\

\clearpage

\subsection{Various Resonant Modes Exist, One is Dependent on Elastic Boundary Thickness d}

Upon investigating resonance regimes, the resonance condition (\ref{resonance}) was found to be satisfied for all D for values of \(K = 1\), \(K = -1\) and \(K = 0\), though the first case is forbidden by (\ref{resonance}). There also exists a value of K satisfying (\ref{resonance}) that varies as the elastic boundary thickness d (or surface flexural rigidity D) is varied. Let us call this mode K'. The dependence of K' on the elastic boundary thickness d is demonstrated below in Figure (\ref{Kmode}).  \\

\begin{figure}[hbt!]
\centering
\includegraphics[scale = .5]{Resonant.png}
\caption{\emph{Flexural-Rigidity-dependent resonant mode K' as a function of elastic boundary thickness d. Note that as d approaches 1 centimeter, the resonant mode K' collides with the aforementioned D-independent mode at \(K = 0\). Here, \(h = 0.5\), \(\sigma = 0\)}} \label{Kmode}
\end{figure}
\\

We can also visualize this relationship by investigating the real roots of the spectral stability function for various values of d in the millimeter regime, and seeing how close the roots are to matching the resonant mode K'. Resonance will not generally occur for non-integer modes, but near-resonances, that is, near-collisions of the mode K' and the real roots of the spectral stability functions can occur for non-integer modes. These near-resonances are depicted below in Figure (\ref{nearres}).
\\

\begin{figure}[hbt!]
\centering
\includegraphics[width=.44\textwidth]{k2.png}\hfill
\includegraphics[width=.44\textwidth]{k6.png}\hfill
\caption{\emph{Spectral Stability Functions for \(d = 6\) mm and \(d = 10 mm\), plotted alongside resonant mode K' for each value of d. Notice the near-collisions of the K' mode with the real roots of the spectral stability functions. Here \(n = 5\), \(h = 0.5\), and \(\sigma = 0\)}} \label{nearres}
\end{figure}
\\

\clearpage


We now focus our attention on the stability of our system under some special cases of the parameters D, \(\sigma\), h, and n. 

\subsection{n = 2 Yields Total Instability for All Parameter Ranges}
\\

For all values of D, \(\sigma\), and h, it can be observed that there exists a real root in the interval (\ref{interval2}) for \((n = 2)\), and thus the system can have instabilities here for all values of D, \(\sigma\), and h. This real root occurs at \(\mu = -1\), which also happens to satisfy the resonance condition (\ref{resonance}) for all values of D, \(\sigma\), and h. Examples of the instabilities at n = 2 are demonstrated in Figure (\ref{n=2}). 
\\

\begin{figure}[hbt!]
\centering
\includegraphics[width=.3\textwidth]{special1.png}\hfill
\includegraphics[width=.3\textwidth]{special2.png}\hfill
\includegraphics[width=.3\textwidth]{special3.png}\hfill
\caption{\emph{Spectral Stability Functions for n = 2. The left graph has parameters h = 10, \(\sigma = 0\), d = 0.01, \(D = 8.1 \times 10^5\). The middle graph has parameters h = 1, \(\sigma = 7\), d = 0.001, \(D = 0.81\). The right graph has parameters h = 5, \(\sigma = 0\), d = 0.01, \(D = 813\). Note that for all graphs, there exists a root at \(\mu = -1\), and thus all of these parameter configurations can have instabilities. }}
\label{n=2}
\end{figure}
\\

\subsection{n = 1 Yields Total Stability For All Parameter Ranges}

Now, consider \(n = 1\). For all values of D, \(\sigma\), and h, it can be observed that there exists solely two real roots on the boundaries of the interval (\ref{interval2}) for \((n = 1)\). The interval is open though, and thus these roots do not constitute possible instabilities of the system since the Krein condition is not satisfied. Thus, for \(n = 1\), all combinations of parameters D, \(\sigma\), and h yield stable solutions. The real roots occur at \(\mu = -1\) and \(\mu = 0 \), which also happen to satisfy the resonance condition (\ref{resonance}) for all values of D, \(\sigma\), and h. Examples of the stabilities at n = 1 are demonstrated in Figure (\ref{n=1}). 
\\

\begin{figure}[hbt!]
\centering
\includegraphics[width=.3\textwidth]{special4.png}\hfill
\includegraphics[width=.3\textwidth]{special5.png}\hfill
\includegraphics[width=.3\textwidth]{special6.png}\hfill
\caption{\emph{Spectral Stability Functions for n = 1. The left graph has parameters h = 10, \(\sigma = 0\), d = 0.1, \(D = 8.1 \times 10^5\). The middle graph has parameters h = 1, \(\sigma = 7\), d = 0.001, \(D = 0.81\). The right graph has parameters h = 5, \(\sigma = 0\), d = 0.01, \(D = 813\). Note that for all graphs, there exists a root at \(\mu = -1\), and thus all of these parameter configurations are stable. }} \label{n=1}
\label{n=2}
\end{figure}
\\

\clearpage 

\section{Discussion}


\subsection{Varying D, \(\sigma\), and h}

When D is varied to be large, it has drastic effects on the compression of the spectral stability functions since the surface flexural term is multiplied by the highest order term \(\mu ^ 5\), and thus dominates the expression as D becomes even remotely large. Furthermore, since the physical range of D is so large (on the order of \(10^9\)), it makes sense that the surface flexural term dominates in the stability analysis for most physical values of D. It is consistent with the literature that for higher values of the surface flexural rigidity, we obtain more high frequency instabilities in the water-wave system. For example, the increase of instabilities with increasing values of surface-flexural rigidity D is a finding that is similar to the results of the 2D Flexural-Gravity deep water stability analysis previously done by Trichtchenko [\ref{olga1}]. A physical mechanism for this proportionality is currently up for speculation. 
\\
\\


When \(\sigma\) is varied, it can be observed that since physical regimes for this parameter vary from approximately 0 to 7 \(N \cdot m\), it follows that since \(\sigma\) is divided by \(\rho = 1000 \frac{kg}{m^3}\) in the spectral stability functions, the magnitudes for terms containing sigma are always small compared to that of the D and g terms, and thus it makes sense that varying \(\sigma\) produces no effects on the stability. Physically, if we have a non-zero \(\sigma\), this implies the non-existence of an elastic boundary on the water surface, and thus the problem we are studying reduces to a less interesting case of water waves simply moving along the interface between our fluid and air. However, it is of note that as expected, the high frequency instabilities that we are studying do not occur in this regime. 
\\
\\

When h is varied to be in the deep water regime (\(h > 1.33\)), it was noted that varying h does not affect the root behaviour of the spectral stability functions. This makes sense as h only really comes into the spectral stability functions as a scaling term in the hyperbolic tangent function, thus it would be surprising if the root behaviour was affected by varying h since h only affects the scaling of the stability functions and not the intercept behaviour. Since the stability of flexural gravity waves has been studied already in the deep water regime, the majority of our results were computed in the shallow water regime \(h \leq 1.33\) 
\\
\\

\subsection{Varying Elastic Boundary Thickness d and Fourier Mode Difference Parameter n Simultaneously in the Millimeter Regime}

Looking at our comparison of d and n values in the centimeter regime for elastic boundary thickness d, Figure (\ref{CM}) essentially zooms in on the interesting stability regimes in the bottom left quadrant of Figure (\ref{MM}). It is of note that we are no longer within the physical thickness regime of ice on water waves, and thus the stabilities in this thickness regime correspond with physical interpretations of fluid moving under elastic boundaries with thicknesses on the order of millimeters, such as blood vessels. It is of note that we can see the effect we had previously discussed of total stability here at \(n = 1\) for all values of d. 
\\

Here, we can see a regime of stability for n greater than 2 and d less than 6 millimeters. This regime of stability most likely accounts for the fact that when d is this small, the surface flexural rigidity term is comparable in magnitude to the gravity term, and thus does not dominate yet in the spectral stability functions. We previously mentioned that these high-frequency instabilities increase with increasing values of D. As the elastic boundary thickness d increases, the surface-flexural rigidity D term increases to the power of three, and thus starts to dominate in it's contribution to the spectral stability functions, yielding real roots in the interval (\ref{interval2}), and thus instability of the system. In Figure (\ref{CM}), we can see a sort of sloping cut-off curve for the stability where for a given value of n greater than 2, if d is sufficiently large, the high frequency instabilities begin to become possible, with higher values of n instabilities occurring at lower cut-off thresholds for the elastic boundary thickness. The implication of this is that these high-frequency instabilities tend to favour larger differences in wavelengths over smaller differences in wavelengths when comparing the dominant plane wave modes in the Fourier-Floquet expansion (\ref{floquet}) comprising our perturbation. A mathematical explanation for this is that raising the Fourier parameter n causes the spectral stability functions (\ref{p2}) and (\ref{p3}) to exhibit root behaviour that converges closer and closer to the midpoint of the interval (\ref{interval2}) for increasing n. A physical interpretation is purely speculative, but may be due to a sort of constructive interference between the plane wave components of the perturbation that grows with increasing values of n, analogous to how for example, the linear combination of \(sin(x) + sin(2x)\) has a maximum amplitude less than that of \(sin(x) + sin(10x)\). 
\\
\\


\subsection{Varying Elastic Boundary Thickness d and Fourier Mode Difference Parameter  n Simultaneously in the Centimeter Regime}
\\

Note that in the methodology, we showed that the existence of real roots in the interval (\ref{interval2}) is equivalent to satisfaction of the Krein condition. Satisfaction of the Krein condition (\ref{krein2}) is a necessary condition for high-frequency instabilities of the water wave problem, and our findings here in Figure (\ref{CM}) support that notion. Looking at our comparison of d and n values in the centimeter regime for elastic boundary thickness d, it can be seen that for most values of d and n in Figure (\ref{CM}), our system can have instabilities. This is because as the elastic boundary thickness increases slightly, the surface flexural rigidity term increases dramatically to the power of three and has the effect of drastically compressing the spectral stability functions such that for most values of d and n, there exists roots in the interval (\ref{interval2}), and thus the Krein condition is satisfied, and so we can have instability. 
\\

Looking to Figure (\ref{CM}), we can discuss the following result. For sufficiently large d, there exists a bifurcation at the vertical line \(n = 1\) that separates the (n,d) plane into stable and unstable regions. For n less than 1, it is of note that this entails guaranteed stability for elastic boundary thickness d greater than 1 cm for our fixed values of \(h = 0.5\) and \(\sigma = 0\). It is interesting to observe that our system is stable for Fourier mode differences \(n<1\). These stabilities occur in this regime because for these values of n, the spectral stability functions (\ref{p1}), (\ref{p2}),(\ref{p3}), and (\ref{p4}) do not exhibit real roots in the interval (\ref{interval2}), and thus this implies that the Krein instability condition (\ref{krein2}) is not satisfied. Since this condition is not satisfied, it could not be possible for our system to exhibit instabilities for \(n<1\), and so the system is stable. Examining the other regime of this bifurcation where \(n > 1\), we notice that real roots of the spectral stability functions do exist however for \(n>1\), and thus instabilities are possible in this n regime. Looking at the Figure (\ref{CM}), the upper right unstable regime implies that for d greater than 1 cm, instabilities can essentially only be generated by waves travelling with differences in Fourier mode greater than that of our initial \(k = 1\) mode.  
\\

A possible physical explanation for both regions of the bifurcation echoes the sentiments of the section above, where a form of constructive interference correlated with increasing n may be occurring to explain the instabilities for \(n>1\). Furthermore, with decreasing n, one might theorize that the amplitudes of the flexural-gravity waves experience a decrease in this 'spectral' constructive interference, so much so that there exists a threshold \((n = 1)\) such that the system experiences the bifurcation, and the amplitudes for \(n < 1\) tend to zero rather than infinity. 
\\


\subsection{K' Resonant Mode}
\\
\\
Examining our results for the K' resonant mode, it is of note that as the thickness d increases to 1 cm, the resonant mode K' collides with the D-independent mode at \(K = 0\). Note that this is the exact cut off point for the little rectangular regime of instability in the bottom left corner of Figure (\ref{CM}) for the same parameter values of \(h = 0.5\) and \(\sigma = 0\). It is possible that this unstable regime is partially due to the presence of this K' resonant mode for values of d less than 1 cm. It can be noted that for the fixed parameters \( n = 5\), \(\sigma = 0\), and \(h = 0.5\), as one varies d from 1 mm to 6 mm, one can observe near-resonant collisions in the interval (\ref{interval2}) which may cause sort of pseudo-resonant amplitude spiking of the flexural gravity waves, and thus instability in this regime. 

\\
\\

\subsection{Total Stability for \(n = 0,1\)}
For the case where \(n = 1\) and for all values of D, \(\sigma\), and h, we observed that there will always exist roots only at the endpoints of the interval (\ref{interval2}). Since there does not exist real roots actually \emph{inside} this interval, it follows that for all configurations of the parameters where \(n = 1\), our system is stable here. It is curious that perturbations with differences in Fourier mode of one do not create instabilities in this system. Physically, this implies that we always get stability when perturbing our high-frequency zero'th order solution with a perturbation solely composed of two plane waves components with periods differing by one second. This unity is rather artificial and arises from our convenient choice of \(2\pi\) wavelength of our initial mode \(k = \frac{2\pi}{\lambda} = 1\), but it is interesting nonetheless, since we have all values of n in a neighborhood around \(n = 1\) yielding unstable regions as seen in Figure (\ref{CM}), as opposed to the stability regime of the vertical line at \(n = 1\). (I don't see how Krein signatures are always equivalent here for \(n = 1\), they are opposite in the very interval we are interested in no?) 
\\

For the case where \(n = 0\) and for all values of D, \(\sigma\), and h, we find that we will always trivially have stability (\ref{interval2}). Since the open interval collapses to a point at the origin, it follows that for all configurations of the parameters where \(n = 0\), there can not exist roots in the interval since the interval no longer exists, and so the Krein condition can never be satisfied, and thus our system is stable here. This is as expected since our perturbation will have the greatest contribution from the mode \(k = 1\) which we chose from the outset, and so we expect that perturbation plane waves with mode equal to \(k = 1\) (and thus \(n=k-k=0\)) would always be stable since our system starts off as stable. 
\\
\\


\subsection{Total Instability for \(n = 2\) and Resonant Effects}
For the case where \(n = 2\) and for all values of D, \(\sigma\), and h, we observed that there will always exist roots at the midpoint of the interval (\ref{interval2}), and that this root happens to also satisfy the resonance condition. Therefore, not only can this system undergo the instabilities we've been studying so far, but it can also cause our system to experience resonance, which can cause amplitudes to grow dramatically and potentially break the ice. To give a physical example, suppose our perturbation has a period \(\mu\) of 1, and that our perturbation is of the simplest form such that it satisfies n = 2, that is, a single plane wave with m = -2, so \( k_{p} = \mu + m = 1 - 2 = -1\). Thus noting that our zero solution has \( k = 1\), we have \(n = |k - k_{p}| = 2\) as required, and our perturbation term \(q^{(1)}\) can be written using (\ref{q2}) as:

\begin{equation} \label{q3}
  q^{(1)}(x,t) = (b_{-2})\delta_{0}e^{\lambda t}e^{-i(x-V_{0}t)}
\\
\end{equation} 

so that using our expression for the linear perturbation velocity (\ref{cp}) with \(k_p = -1\), we recover:

\begin{equation} \label{cp}
  c_{p} = \pm \frac{\sqrt{(g(-1) - \frac{\sigma(-1)^3}{\rho} + \frac{D(-1)^5}{\rho})\tanh{(-h)}}}{-1} 
\\
\end{equation} 

which implies (taking the negative sign corresponding to propagation of the load in the negative \(\hat{x}\) direction):

\begin{equation} \label{cp2}
  c_{p} = \sqrt{(g - \frac{\sigma }{\rho} + \frac{D }{\rho})\tanh{(h)}} = V_{0}
\\
\end{equation} 

but this is the exact expression we found for the wave speed (\ref{V}), so that means that \(|c_{p}| = |V_0|\) and thus the resonance condition (\ref{resonance2}) is satisfied. Therefore, our perturbation for q is specified here such that the flexural-gravity waves undergo high-frequency instabilities that may also break the ice at critical amplitudes via the resonance method of ice destruction.
\\
\\


\section{Conclusion}

In summation, upon variation of the parameters D, \(\sigma\), and h individually, while holding the others constant, we found that there exists stability regimes in shallow water only under variation of the surface flexural rigidity D, and not the surface tension \(\sigma\). We found that as D increases, so do the number of eigenvalue collisions representing real roots in the Krein interval, and thus instabilities of the system. The elastic boundary thickness d and the Fourier mode difference parameter n were varied simultaneously in the millimeter and centimeter regimes of d. In the centimeter regime, a bifurcation at \(n = 1\) was found such that for sufficiently large d, there exists a regime of stability for all \(n<1\) and a regime of instability for all \(n>1\). In the millimeter range for d, we found a regime of stability for \(n>2\) and sufficiently low elastic boundary thickness, with the instability cutoff thickness decreasing with increasing n. We found a regime of instability for \(0 < n < 1\), and sufficiently low elastic boundary thickness d. We also found a resonant mode K' that decreases with increasing elastic boundary thickness d and exhibits near-resonant high-frequency instabilities if K' is in the open interval prescribed by the Krein signature condition. We also found regimes of total stability under variation of all parameters for \(n = 0\), \(n = 1\), as well as total instability under variation of all parameters for \(n = 2\), with a real root in the Krein interval that always satisfies the resonance conditions (\ref{resonance}) and (\ref{resonance2}), and thus is an example in our system of the phenomena of the resonance method of ice destruction. 
\\

\section{Acknowledgements}
\\ 
Many thanks to Dr. Katie Oliveras, Associate Professor of Mathematics at Seattle University for her guidance with technical details of the AFM formulation of the water-wave problem. Thanks to Bernard Deconninck, J.-M. Vanden-Broeck, P. Milewski and E. Parau for their illuminating findings in their respective resesarch in the stability of water waves. Lastly, I'd like to express my gratitude to Dr. Olga Trichtchenko and Dr. Giovanni Fanchini for their continued support and guidance throughout the term -- this was one of my favourite experiences in my undergraduate degree.

\clearpage

\subsection{References}
\\
\begin{enumerate}

\item Trichtchenko, Olga, et al. “Stability of Periodic Travelling Flexural-Gravity Waves in Two Dimensions.” ArXiv.org, 29 Jan. 2018, arxiv.org/abs/1801.09748. \label{olga1} \\

\item Trichtchenko, Olga. “A Method for Identifying Stability Regimes Using Roots of a Reduced-Order Polynomial.” SpringerLink, Birkhäuser, Cham, 1 Jan. 1970, \(link.springer.com/chapter/10.1007/978-3-030-33536-6_12\).\label{reduced}\\

\item Nonlinear Dynamics and Chaos: with Applications to Physics, Biology, Chemistry, and Engineering, by Steven H. Strogatz, CRC Press, 2019.  \label{stabilitypde}

\item Yirka, Bob. “Study Suggests Large Waves May Have Bigger Role in Breaking up Polar Sea Ice than Thought.” Phys.org, Phys.org, 29 May 2014, phys.org/news/2014-05-large-bigger-role-polar-sea.html. \label{icewavespic}

\item “Vasculature of the Heart.” TeachMeAnatomy, teachmeanatomy.info/thorax/organs/heart/heart-vasculature/.  \label{heart}

\item Deconinck, Bernard, and Katie Oliveras. “The Instability of Periodic Surface Gravity Waves.” Journal of Fluid Mechanics, vol. 675, 2011, pp. 141–167., doi:10.1017/s0022112011000073. \label{katie}

\item Ablowitz, M. J., et al. “On a New Non-Local Formulation of Water Waves.” Journal of Fluid Mechanics, vol. 562, 2006, p. 313., doi:10.1017/s0022112006001091. \label{AFM}

\item \label{bandstructure} Chou, Tom. “Band Structure of Surface Flexural–Gravity Waves along Periodic Interfaces: Journal of Fluid Mechanics.” Cambridge Core, Cambridge University Press, 25 Aug. 1998, www.cambridge.org/core/journals/journal-of-fluid-mechanics/article/band-structure-of-surface-flexuralgravity-waves-along-periodic-interfaces/8C78D8086BB5013F0340D5C970B416E7. \\

\item Merle, F., and L. Vega. “L 2 Stability of Solitons for KdV Equation.” OUP Academic, Oxford University Press, 1 Jan. 2003, academic.oup.com/imrn/article-abstract/2003/13/735/681769?redirectedFrom=fulltext. \\

\item Bona, Jerry L., and Robert L. Sachs. “Global Existence of Smooth Solutions and Stability of Solitary Waves for a Generalized Boussinesq Equation.” Communications in Mathematical Physics, Springer-Verlag, projecteuclid.org/euclid.cmp/1104161906. \\


\item Gao, Tao, et al. “Numerical Computations of Two-Dimensional Flexural-Gravity Solitary Waves on Water of Arbitrary Depth.” OUP Academic, Oxford University Press, 22 Mar. 2018, academic.oup.com/imamat/article-abstract/83/3/436/4951679. \\

\item Bryant, P. J. “Stability of Periodic Waves in Shallow Water: Journal of Fluid Mechanics.” Cambridge Core, Cambridge University Press, 29 Mar. 2006, www.cambridge.org/core/journals/journal-of-fluid-mechanics/article/stability-of-periodic-waves-in-shallow-water/206CD291B43721C4D7C949AA6FEFA251. \\

\item Hammack, Joe, et al. “Two-Dimensional Periodic Waves in Shallow Water. Part 2. Asymmetric Waves: Journal of Fluid Mechanics.” Cambridge Core, Cambridge University Press, 26 Apr. 2006, www.cambridge.org/core/journals/journal-of-fluid-mechanics/article/twodimensional-periodic-waves-in-shallow-water-part-2-asymmetric-waves/9DAC2118F4C9C681143EFAA1E41BDF8A. \\

\end{enumerate}

\end{document}
